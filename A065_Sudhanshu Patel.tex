\documentclass{article}
\usepackage[pdftex]{graphicx}
% Comment the following line to NOT allow the usage of umlauts
\usepackage[utf8]{inputenc}
% Uncomment the following line to allow the usage of graphics (.png, .jpg)
%\usepackage{graphicx}

% Start the document
\begin{document}

% Create a new 1st level heading
\section{"Emerging Technologies in Healthcare"} 

1. INTRODUCTION:\\ 
\\
Emerging technologies are technologies whose development, practical applications, or both are still largely unrealized, such that they are figuratively emerging into prominence from a background of nonexistence or obscurity.\\

Health technologies comprise of all the devices, medicines, vaccines, processes, procedures, and systems designed to streamline 
healthcare operations, lower costs, and enhance quality of care. Technology drives healthcare more than any other force. It is 
drastically changing and improving healthcare, from anesthetics and antibiotics to MRI scanners and radiotherapy. This 
technology-driven progress in healthcare is often called Health 2.0. It is well known that hospitals adopt new technologies that 
enhance their service capabilities and enable them to attract and retain physicians who use the technologies.
\\
\\
2.CONCEPT OF EMERGING TECHNOLOGY:\\

Emerging technology (ET) lacks a consensus on what classifies them as “emergent.” It is a relative term because one may see a 
technology as emerging and others may not see it the same way. It is a term that is often used to describe a new technology. A 
technology is still emerging if it is not yet a “must-have” . An emerging technology is the one that holds the promise of creating 
a new economic engine and is trans-industrial.\\ 
\\
ET is used in different areas such as media, healthcare, business, science, or education. Emerging healthcare technologies cannot 
be fully exploited without a clinical team to shape the therapeutic response, something hospitals have been able to do over the 
years with their multidisciplinary clinical workforce. How hospitals and policymakers respond to these emerging technologies will 
help determine whether hospitals remain at the center of the US healthcare system. Some US hospitals have remarkably responded 
to these new technologies and adapted their service offerings to incorporate new technologies.
\\
\\
3.EMERGING HEALTHCARE TECHNOLOGIES:\\
\\
Emerging technologies in healthcare include information technology, nanotechnology/nanomedicine, biotechnology, cloud 
computing, Internet of things, augmented/virtual reality, GPS, RFID, voice search, chatbots, social media, blockchain, 
personalized medicine, biometrics, electronic health records, wearable computing devices, drones, robotics, and artificial 
intelligence. Of the several emerging technologies, the following examples stand out:\\
A) Artificial intelligence (AI):\\
 This is a field of computer science that is concerned with designing systems to do things that 
would require intelligence of humans. Today, artificial intelligence is shorthand for any task a machine can perform just as 
well as, if not better than, humans. Today, artificial intelligence is shorthand for any task a machine can perform just as 
well as, if not better than, humans. AI in healthcare refers to the application of AI technology in
 the diagnosis and treatment of patients. AI is being applied in healthcare to review mammograms, monitor early stage heart
 disease, and enable accurate decision-making among medical practitioners. \\
B) Robotics:\\
This deals with the design, construction, operation, and application of robots. Robots are becoming an integral part 
of the healthcare toolkit. Robots
play an important role in healthcare as they can improve diagnosis, lower the number of medical errors, and improve the 
overall quality and effectiveness of healthcare delivery. They hold the promise of addressing major healthcare issues in 
surgery, diagnostics, prosthetics, physical and mental therapy, monitoring, and support.

\includegraphics[scale=0.1]
{full.jpg}



C). BLOCKCHAIN:\\ 
This technology consists of a shared or distributed database used to maintain a growing list of transactions, called 
blocks. With blockchain (BC), transaction records are stored and distributed across all network participants rather than at a 
central location. Blockchain in health care will be in clinical trial records, regulatory compliance, and medical records. 

Blockchain is able to securely, privately and comprehensively track patient health records. It makes electronic medical 
records more efficient, disintermediated, and secure. It also makes health information exchanges (HIE) more secure, efficient, 
and interoperable.\\

4.CONCLUSION:\\
Frankly speaking, healthcare has no end of problems: we all want and expect better care, costs are rising, and performance is 
declining, we live living longer with chronic illness, etc. If we want healthcare to improve in the future, we must continuously 
plan for it.
Future technological innovations (new drugs, new treatments, new devices, etc.) will keep transforming healthcare. Since 
technology drives healthcare, the fundamental problems of wellbeing, health and happiness, will remain. We need to be aware of 
the drivers, align with them, and work with them to ensure the best outcomes for society. 




% Uncomment the following two lines if you want to have a bibliography
%\bibliographystyle{alpha}
%\bibliography{document}

\end{document}
